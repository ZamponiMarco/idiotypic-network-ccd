\section{Intro}

Immunity or resistance is the ability to use one's physical defenses to counter damage or disease. The two types of immunity are: 

\begin{enumerate}
    \item innate;
    \item adaptive.
\end{enumerate}

For this project we will focus on adaptive immunity. Adaptive (or acquired, specific) immunity refers to the defenses that involve the recognition of a pathogen once it has managed to overcome the defenses of innate immunity. It provides a targeted response to various microorganisms, treating them in a differentiated way. Unlike innate immunity, adaptive immunity is slower in response, has a fundamental memory component and is implemented by T and B lymphocytes and macrophages.

\subsection{Adaptive Immunity}
Adaptive immunity involves the production of specific cells or antibodies to destroy a particular antigen. An antigen is defined as any element, such as pathogens, food, drugs, pollen, or tissue, that the immune system recognizes as foreign.

\subsection{Maturation of B and T lymphocytes}
The cells that carry the adaptive immune response are B and T lymphocytes. Both develop from stem cells originating in the red bone marrow, but as the B lymphocytes complete their development in the red bone marrow, the T lymphocytes migrate to the thymus, where they will complete. their maturation process. Then before leaving their respective maturation sites, T and B lymphocytes begin to produce different membrane specific proteins. Among these, some act as antigen receptors, i.e. molecules capable of recognizing and binding to specific antigens

\subsection{Different types adaptive immunity response}

Adaptive immunity consists of two types of closely related immune responses, both of which are triggered by antigens.

In the cell-mediated response some T cells behave like an army of soldiers directly attacking the invading antigen by chemical and physical means.

In the antibody-mediated response, B lymphocytes are transformed into plasma cells that synthesize and secrete antibodies; a given antibody can bind and inactivate a specific antigen.

Other T lymphocytes participate in both cell-mediated and antibody-mediated immune responses. Although each type of response is specific to address the different aspects of an invasion, a given pathogen can trigger both.

\subsection{antigens and antibodies}

An antigen triggers the production of specific antibodies and / or T lymphocytes in the body. Complete microorganisms or parts of them, components of bacterial structures, bacterial toxins and viral proteins can act as antigens. Other examples of antigens could be the chemical components of pollen, egg white, incompatible blood cells, and transplanted tissues or organs.

\subsection{Self and non-self antigens}

Self antigens are found on the surface of the plasma membrane of most of the body's cells and serve as markers of cellular identity, their main function being to help T lymphocytes recognize whether an antigen is foreign or not.

\subsection{Antibodies or immunoglobins}
The antigens cause the plasma cells to secrete proteins known as antibodies. At the ends of these antibodies are variable regions which are in fact the antigen binding sites, ie the parts of an antibody that "match" and bind to an antigen. Antibodies belong to a group of plasma proteins also known as immunoglobins.

\pagebreak